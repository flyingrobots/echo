% SPDX-License-Identifier: Apache-2.0 OR MIND-UCAL-1.0
% © James Ross Ω FLYING•ROBOTS <https://github.com/flyingrobots>

\chapter{Tour de Code: Overview}
\label{chap:tour-overview}

\begin{quote}
\textbf{The complete function-by-function trace of Echo's execution pipeline.}

This tour traces every function call involved in processing a user action through the Echo engine. File paths and line numbers are accurate as of 2026-01-18.
\end{quote}

\begin{commentary}
Welcome to the Tour de Code. I'm going to walk you through this codebase like we're pair programming. When I see something clever, I'll tell you why it's clever. When there's a non-obvious design decision, I'll explain the trade-offs.

The goal: by the end of this tour, you'll understand not just \emph{what} the code does, but \emph{why} it's structured the way it is.
\end{commentary}

\section{The Pipeline at a Glance}

A user action flows through Echo in eight stages:

\begin{enumerate}
    \item \textbf{Intent Ingestion} --- User action becomes an \texttt{Intent}
    \item \textbf{Transaction Lifecycle} --- Intents are batched into transactions
    \item \textbf{Rule Matching} --- Rules produce \texttt{PendingRewrite}s
    \item \textbf{Scheduler: Drain \& Reserve} --- Radix sort + independence check
    \item \textbf{BOAW Parallel Execution} --- Workers execute in parallel
    \item \textbf{Delta Merge} --- Thread-local deltas merge canonically
    \item \textbf{Hash Computation} --- State root, patch digest, commit hash
    \item \textbf{Commit Orchestration} --- Final state materialization
\end{enumerate}

\begin{bigpicture}
Every stage is designed around one principle: \textbf{determinism}. Given the same inputs, the engine must produce identical outputs regardless of thread scheduling, CPU count, or platform.

This isn't just about correctness---it enables replay, forking, and distributed consensus.
\end{bigpicture}

\section{Key Files}

\begin{tabular}{ll}
\toprule
\textbf{Component} & \textbf{Primary File} \\
\midrule
Engine core & \texttt{crates/echo-core/src/engine.rs} \\
Transaction handling & \texttt{crates/echo-core/src/transaction.rs} \\
Rule matching & \texttt{crates/warp-core/src/rule\_matching.rs} \\
Scheduler (Radix) & \texttt{crates/warp-core/src/radix\_scheduler.rs} \\
BOAW execution & \texttt{crates/warp-core/src/boaw/exec.rs} \\
Shard routing & \texttt{crates/warp-core/src/boaw/shard.rs} \\
Delta/merge & \texttt{crates/warp-core/src/tick\_delta.rs} \\
Hashing & \texttt{crates/warp-core/src/hashing.rs} \\
\bottomrule
\end{tabular}

\section{Reading This Tour}

Each section follows a consistent format:

\begin{itemize}
    \item \textbf{Entry Point} --- The function that kicks off this stage
    \item \textbf{Call Trace} --- Indented tree showing the execution flow
    \item \textbf{Data Structures} --- Key types involved
    \item \textbf{Commentary} --- Why it works this way
\end{itemize}

\begin{protip}
Keep a terminal open with the codebase. When you see a file reference like \texttt{src/boaw/exec.rs:61-83}, jump to that location and read the actual code alongside this tour.
\end{protip}
