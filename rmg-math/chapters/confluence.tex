\section{Confluence and Two-Plane Commutation (DPOI)}
\label{sec:confluence}
\paragraph{Setting.}
Fix a type set $T$. $\GraphT$ is the category of $T$-typed directed graphs; $\OGraphT$ is the adhesive category of typed open graphs (cospans $I\to G\leftarrow O$ with monos) \cite{LackSobocinski2006Adhesive}.
Rules and DPO rewriting follow the standard treatment \cite{EhrigLowe1997DPO,Ehrig2006FAGT}.
A \emph{DPOI rule} is a span of monos $p=(L \xleftarrow{\ell} K \xrightarrow{r} R)$; a \emph{match} is a boundary-preserving mono $m:L\mono G$ satisfying gluing (dangling/identification). The step $G\To_p H$ is given by the standard double square:
\DPO{K}{L}{R}{D}{G}{H}
Typed ports are enforced by boundary typing; if violated, the pushout complement does not exist.

\paragraph{RMG two-plane state.}
An RMG state is $(G;\alpha,\beta)$ with skeleton $G\in\OGraphT$ and attachments $\alpha(v)$, $\beta(e)$ in fibers over nodes/edges. A \emph{tick} applies attachment steps (fibers) \emph{then} skeleton steps (base), under the invariant \textbf{no-delete-under-descent} (and a ``no-clone'' policy on preserved items).

\paragraph{Scheduler independence.}
For $m:L\mono G$ of $p=(L\leftarrow K\to R)$, let $\Del(m)=m(L\setminus K)$ and $\Use(m)=m(L)$. Two matches $m_1,m_2$ are \emph{parallel independent} iff
$\Del(m_1)\cap \Use(m_2)=\varnothing$ and $\Del(m_2)\cap \Use(m_1)=\varnothing$, and both satisfy gluing.
Operationally we use a safe over-approximation, the \emph{touch set} $\Use(m)\cup \Halo_r(\Use(m))$ (kernel radius $r$), and select a maximal independent set (MIS).

\begin{theorem}[Tick-level confluence (Theorem A)]
Given a scheduler-admissible batch (pairwise parallel independent in the base; attachments under no-delete-under-descent), applying the batch in any serial order consistent with attachments-first yields a unique result up to typed open-graph isomorphism.
\end{theorem}
\begin{proof}[Sketch]
By the Concurrency/Parallel Independence Theorem for DPO in adhesive categories, independent base steps commute. Attachment steps commute in the product of fibers; applied first, they are unaffected by base updates.
\end{proof}

\begin{theorem}[Two-plane commutation (Theorem B)]
Under no-delete-under-descent, performing all attachment updates then base updates equals (up to iso) performing base updates then transporting and applying the attachment updates in the new fibers.
\end{theorem}
\begin{proof}[Sketch]
Base updates are pushouts along monos in $\OGraphT$. Reindexing along base monos preserves pushouts in fibers (Van Kampen). Hence the square ``attachments vs base'' commutes up to isomorphism.
\end{proof}

\begin{theorem}[Conditional global confluence]
Let $R$ be a finite DPOI rule set. If all DPOI critical pairs are joinable (modulo boundary iso) and rewriting terminates (or admits a decreasing-diagrams labelling), then $\Rightarrow_R$ is confluent.
\end{theorem}
\begin{proof}[Idea]
Critical Pair Lemma $\Rightarrow$ local confluence; combine with Newman's Lemma (or Decreasing Diagrams) for global confluence.
\end{proof}

\paragraph{Math-to-code contract.}
\emph{Independence check}: require $\Del(m_1)\cap \Use(m_2)=\varnothing$ and symmetric, plus gluing.
\emph{Two-plane discipline}: forbid delete/clone of any position touched in fibers; publish attachments before skeleton.
