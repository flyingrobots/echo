\documentclass[11pt]{article}
\usepackage[a4paper,margin=1in]{geometry}
\usepackage{microtype}
\usepackage{mathtools}
\usepackage{sty/rmg-macros}
% rmg-diagrams.tex — tikz-cd helpers for DPO squares
\usepackage{tikz-cd}
\tikzcdset{row sep/normal=large, column sep/normal=large}

% Double-pushout template:
% \DPO{K}{L}{R}{D}{G}{H}
\newcommand{\DPO}[6]{%
\begin{tikzcd}
#1 \arrow[r, hook] \arrow[d, hook] & #2 \arrow[d, hook] & \qquad
#1 \arrow[r, hook] \arrow[d, hook] & #3 \arrow[d, hook] \\
#4 \arrow[r, hook] & #5 & \qquad #4 \arrow[r, hook] & #6
\end{tikzcd}%
}


\title{Recursive Metagraphs (RMG): DPOI Semantics, Confluence, Hypergraph Embedding, and Rulial Distance}
\author{James Ross • RMG Core • Project "Echo"}
\date{\today}

\begin{document}
\maketitle

\begin{abstract}
We formalize the execution model of Recursive Metagraphs (RMG) using Double-Pushout with Interfaces (DPOI) in the adhesive category of typed open graphs.
We prove \emph{tick-level confluence} (deterministic batches) and \emph{two-plane commutation} (attachments-first is correct), and give conditions for \emph{global confluence} via critical pairs.
We provide a faithful incidence encoding of typed open hypergraphs into our setting, preserving DPO steps and lifting multiway derivations, and define a task- and resource-aware \emph{rulial distance} as an MDL-based pseudometric over observers.
\end{abstract}

\tableofcontents

\section{Notation and Setting}
Typed open graphs $\OGraphT$ (cospans of monos) form an adhesive category. Rules are linear spans $(L\leftarrow K\to R)$; matches are boundary-preserving monos satisfying gluing. RMG states $(G;\alpha,\beta)$ carry attachments in the fibers over nodes and edges; publishing is two-plane: attachments then skeleton.

\section{Confluence and Two-Plane Commutation (DPOI)}
\label{sec:confluence}
\paragraph{Setting.}
Fix a type set $T$. $\GraphT$ is the category of $T$-typed directed graphs; $\OGraphT$ is the adhesive category of typed open graphs (cospans $I\to G\leftarrow O$ with monos) \cite{LackSobocinski2006Adhesive}.
Rules and DPO rewriting follow the standard treatment \cite{EhrigLowe1997DPO,Ehrig2006FAGT}.
A \emph{DPOI rule} is a span of monos $p=(L \xleftarrow{\ell} K \xrightarrow{r} R)$; a \emph{match} is a boundary-preserving mono $m:L\mono G$ satisfying gluing (dangling/identification). The step $G\To_p H$ is given by the standard double square:
\DPO{K}{L}{R}{D}{G}{H}
Typed ports are enforced by boundary typing; if violated, the pushout complement does not exist.

\paragraph{RMG two-plane state.}
An RMG state is $(G;\alpha,\beta)$ with skeleton $G\in\OGraphT$ and attachments $\alpha(v)$, $\beta(e)$ in fibers over nodes/edges. A \emph{tick} applies attachment steps (fibers) \emph{then} skeleton steps (base), under the invariant \textbf{no-delete-under-descent} (and a ``no-clone'' policy on preserved items).

\paragraph{Scheduler independence.}
For $m:L\mono G$ of $p=(L\leftarrow K\to R)$, let $\Del(m)=m(L\setminus K)$ and $\Use(m)=m(L)$. Two matches $m_1,m_2$ are \emph{parallel independent} iff
$\Del(m_1)\cap \Use(m_2)=\varnothing$ and $\Del(m_2)\cap \Use(m_1)=\varnothing$, and both satisfy gluing.
Operationally we use a safe over-approximation, the \emph{touch set} $\Use(m)\cup \Halo_r(\Use(m))$ (kernel radius $r$), and select a maximal independent set (MIS).

\begin{theorem}[Tick-level confluence (Theorem A)]
Given a scheduler-admissible batch (pairwise parallel independent in the base; attachments under no-delete-under-descent), applying the batch in any serial order consistent with attachments-first yields a unique result up to typed open-graph isomorphism.
\end{theorem}
\begin{proof}[Sketch]
By the Concurrency/Parallel Independence Theorem for DPO in adhesive categories, independent base steps commute. Attachment steps commute in the product of fibers; applied first, they are unaffected by base updates.
\end{proof}

\begin{theorem}[Two-plane commutation (Theorem B)]
Under no-delete-under-descent, performing all attachment updates then base updates equals (up to iso) performing base updates then transporting and applying the attachment updates in the new fibers.
\end{theorem}
\begin{proof}[Sketch]
Base updates are pushouts along monos in $\OGraphT$. Reindexing along base monos preserves pushouts in fibers (Van Kampen). Hence the square ``attachments vs base'' commutes up to isomorphism.
\end{proof}

\begin{theorem}[Conditional global confluence]
Let $R$ be a finite DPOI rule set. If all DPOI critical pairs are joinable (modulo boundary iso) and rewriting terminates (or admits a decreasing-diagrams labelling), then $\Rightarrow_R$ is confluent.
\end{theorem}
\begin{proof}[Idea]
Critical Pair Lemma $\Rightarrow$ local confluence; combine with Newman's Lemma (or Decreasing Diagrams) for global confluence.
\end{proof}

\paragraph{Math-to-code contract.}
\emph{Independence check}: require $\Del(m_1)\cap \Use(m_2)=\varnothing$ and symmetric, plus gluing.
\emph{Two-plane discipline}: forbid delete/clone of any position touched in fibers; publish attachments before skeleton.

\input{chapters/embedding.tex}
\section{Rulial Distance as a Pseudometric via MDL Translators}
\label{sec:rulial-distance}
Fix an RMG universe $(U,R)$ and its history category $\mathrm{Hist}(U,R)$.
An \emph{observer} is a boundary-preserving functor $O:\mathrm{Hist}(U,R)\to \mathcal{Y}$ (symbol streams or causal-annotated traces) under budgets $(\tau,m)$.
A \emph{translator} $T:O_1\Rightarrow O_2$ is an open-graph transducer (small DPOI rule pack) with $O_2\approx T\circ O_1$.

Let $\mathrm{DL}(T)$ be a prefix-code description length (MDL) and let $\mathrm{Dist}$ be a task-appropriate distortion on outputs.
Define the symmetric distance
\[
D^{(\tau,m)}(O_1,O_2)=\inf_{T_{12},T_{21}}\ \mathrm{DL}(T_{12})+\mathrm{DL}(T_{21})
+\lambda\big(\mathrm{Dist}(O_2,T_{12}\circ O_1)+\mathrm{Dist}(O_1,T_{21}\circ O_2)\big).
\]
Assume $\mathrm{DL}$ is subadditive up to a constant $c$ and $\mathrm{Dist}$ is a metric/pseudometric.

\begin{proposition}[Pseudometric]
$D^{(\tau,m)}$ is a pseudometric (nonnegative, symmetric, $D(O,O)=0$).
\end{proposition}

\begin{theorem}[Triangle inequality]
$D^{(\tau,m)}(O_1,O_3)\le D^{(\tau,m)}(O_1,O_2)+D^{(\tau,m)}(O_2,O_3)+2c$.
\end{theorem}
\begin{proof}[Sketch]
Choose near-minimizers for the two terms; compose translators: $T_{13}=T_{23}\circ T_{12}$ and $T_{31}=T_{21}\circ T_{32}$. Subadditivity of $\mathrm{DL}$ and the metric triangle for $\mathrm{Dist}$ bound the composed cost; take infima.
\end{proof}

\paragraph{Operational estimator.}
Compile translators as DPOI rule packs; measure $\mathrm{DL}$ by compressed bundle size and $\mathrm{Dist}$ on a fixed test suite under resource budgets. This yields an empirical (approximate) $D^{(\tau,m)}$.

\appendix
\section{Scheduler Contract: Math $\leftrightarrow$ Engine}
\label{app:scheduler}
For a compiled rule $p=(L\leftarrow K\to R)$ and match $m$:
\begin{itemize}
\item $\Del(m)=\im(L\setminus K)$; $\Use(m)=\im(L)$.
\item Independence requires $\Del(m_1)\cap\Use(m_2)=\varnothing$ and symmetrically, plus gluing.
\item The scheduler computes an MIS over $\mathrm{Touch}(m)=\Use(m)\cup\Halo_r(\Use(m))$.
\item Two-plane: if a fiber update touches $\alpha(v)$ or $\beta(e)$, no concurrent base step may delete/clone $v$ or $e$; publish attachments, then base.
\end{itemize}


% References
\nocite{Ehrig2006FAGT,LackSobocinski2004Adhesive,LackSobocinski2006Adhesive,EhrigLowe1997DPO,vanOostrom1994Decreasing,Fong2015DecoratedCospans,HabelPlump2002Relabelling}
\bibliographystyle{alpha}
\bibliography{refs}

\end{document}
